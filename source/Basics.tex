%
% Basics.tex
% Basics
%
% R for Business Administration
%
% Copyright (C) 2016 Harris Kenny, Brandon DeGolier, Nate Lewis
%
% This document is licensed under the Creative Commons Attribution 4.0
% International Public License (CC BY-SA 4.0)
%

\section{User Interface}

\subsection{R}
R has a commmand-line interface, however multiple projects have developed a graphical user interface (GUI) for R. A GUI is not required to use R and these packages may still require the user to use the command-line interface for specific functions.

% Need screen shot of R command-line interface

\subsection{R-Commander}
The R-Commander GUI is a simple additional window that contains commonly structured menus, buttons, information fields, and dialogue boxes. It also contains script and output text windows. Commands generated through dialogs will automatically post to the output window with their corresponding printed output, and also post to the script window. Lines in the script menu can be edited and resubmitted. Finally, errors, warnings, and notes appear in the messages window at the bottom of the interface.

% Need screen shot of R-Commander interface

\section{Datasets}

\subsection{Importing Data}
Using Rcmdr, select Data then drop down to Import Data to import a dataset based (categorized by file format). If you are using R Script, read.table is the most common command to do this. There are many compatible file formats and ways to import data into R, find support for specific common file types below, or online.

Importing CSV Files

Using R Script and read.table, note whether your CSV has a header or not. If it does, mark TRUE. If not, mark FALSE.

> Dataset <-  read.table("/filename.csv", header = FALSE, sep = ",")

Importing ODS Files

It is possible to import ODS file (an Open Document Format spreadsheet) directly into R after installing a separate package called readODS. Once installed, this is an example of an R Script command to import an ODS file:

> read.ods(file = NULL, sheet = NULL, formulaAsFormula = FALSE)

There are several arguments you can use with this command, find a more detailed walkthrough online here: \texttt{https://cran.r-project.org/web/packages/readODS/readODS.pdf}

\subsection{Viewing Datasets}
There are multiple ways to view data you have uploaded into R. This is valuable because it is a way to confirm that you properly imported the data. You should always verify that the data imported as you intended before running commands.

View dataset
>print(Dataset)

View number of records in dataset
>nrow(Dataset)

View variable names in dataset
>names(Dataset)

View data types of variables in dataset
>str(Dataset)

View last records in dataset
>tail(Dataset)

Summary statistics for dataset
>summary(Dataset)

\subsection{Active Dataset}
Confirm which is your active dataset in R-Commander with the dataset indicator in the top of the main GUI. You can change the dataset by clicking this window to view a dropdown list. It is important to verify you have the right active data before running commands.

\subsection{Editing Datasets}