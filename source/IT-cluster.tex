%
% IT.
% Information Technology
%
% Aleph Objects Operations Manual
%
% Copyright (C) 2014, 2015 Aleph Objects, Inc.
%
% This document is licensed under the Creative Commons Attribution 4.0
% International Public License (CC BY-SA 4.0) by Aleph Objects, Inc.
%

\section{3D Printer Cluster}
There are 144 printers in the 3D printing cluster. One hundred thirty five are
in the main cluster room, nine in the adjoining sample room. The cluster is a
mix of LulzBot TAZ and LulzBot Minis.

Each printer has a Beaglebone Black (BBB) connected to it via USB. The
BBB is running Debian (armhf port) and Botqueue. There is a separate Botqueue
server, also running Debian, that the BBBs connect to, to get print jobs.

The printers are organized in sets, or ``pods'', typically of nine. Each cabinet
holds nine machines, three wide by three high. Each pod is assigned a letter.
In the main cluster room, this is A through O. In the sample room, it is pods
Y and Z which have five and four machines, respectively. Machines are named
of the format: bbb-a1, bbb-a2, through to bbb-a9 for the first pod. Then the
next pod starts bbb-b1, through to the end: bbb-z4.

List of printers:
\begin{itemize}
\item \texttt{bbb-a1.alephobjects.com} through
      \texttt{bbb-o9.alephobjects.com} --- LulzBot TAZ.
\item \texttt{bbb-y1.alephobjects.com} through
      \texttt{bbb-z4.alephobjects.com} --- LulzBot Mini.
\end{itemize}

Links to upstream:
\begin{itemize}
\item \href{http://beagleboard.org/}{BeagleBone Black}
\item \href{http://botqueue.org/}{BotQueue}
\item \href{https://wiki.debian.org/ArmHardFloatPort}{Debian armhf}
\end{itemize}

