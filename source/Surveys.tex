%
% Surveys.tex
% Surveys
%
% R for Business Administration
%
% Copyright (C) 2016 Harris Kenny, Brandon DeGolier, Nate Lewis
%
% This document is licensed under the Creative Commons Attribution 4.0
% International Public License (CC BY-SA 4.0)
%

\section{Types of Surveys}
Surveys are a valuable way to learn stakeholders' perceptions of your organization's past, present, and future performance. Below are examples of survey groups that may be relevant for your organization:

\begin{itemize}
 \item Advisors
 \item Channel Partners
 \item Customers
 \item Developers
 \item Employees
 \item Journalists
 \item Investors
 \item Policymakers
 \item Suppliers
 \item Students
 \item Tradeshow Attendees
\end{itemize}

Let's try exercises using categorical data that could be collected through surveys. 

\section{Proportion Tests}
Use proportion tests to compare the proportion or mean from one group to a specified value. These will generate a p-value that can be used to evaluate the statistical significance of the comparison.\footnote{Learn more at: \texttt{https://stat.ethz.ch/R-manual/R-devel/library/stats/html/prop.test.html}.}

The exercises in this section will use the "warpbreaks" dataset (The Number of Breaks in Yarn during Weaving) that is included with R. Load this data set into R following the instructions outlined in the Installation section of this work.

\subsection{One Sample Proportion Test}
prop.test(<count>, <total>)
prop.test(384,5000)
prop.test(384, 5000, conf.level=0.95)

\subsection{Two Sample Proportion Test}
prop.test(x = c(388, 545), n = c(443, 564), correct = FALSE)

2-sample test for equality of proportions without continuity
        correction

data:  c(388, 545) out of c(443, 564)
X-squared = 29.824, df = 1, p-value = 4.731e-08
alternative hypothesis: two.sided
95 percent confidence interval:
 -0.12459255 -0.05633856
sample estimates:
   prop 1    prop 2 
0.8758465 0.9663121 

\section{Generating Metadata from Open Responses}
It is possible to create metadata from open-ended responses in survey questions that you evaluate through statistical techniques.

For example, say a survey asked customers if there is anything your organization could do to better serve them in the future. Have someone from your organization review the responses, generate a list based on topics submitted by customers. 

Next, count each time on appears in the customer responses. This can be done manually or through a formula in spreadsheet software like LibreOffice Calc or Gnumeric.\footnote{Learn more about these projects at \texttt{https://www.libreoffice.org/} and \texttt{http://gnumeric.org/}} This new set of metadata can now be leveraged in proportion tests or other testing methods.