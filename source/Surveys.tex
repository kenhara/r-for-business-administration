%
% Surveys.tex
% Surveys
%
% R for Business Administration
%
% Copyright (C) 2016 Harris Kenny, Brandon DeGolier, Nate Lewis
%
% This document is licensed under the Creative Commons Attribution 4.0
% International Public License (CC BY-SA 4.0)
%

\section{Types of Surveys}
Surveys are a valuable way to learn stakeholders' perceptions of your organization's past, present, and future performance. Below are examples of survey groups that may be relevant for your organization:

\begin{itemize}
 \item Advisors
 \item Channel Partners
 \item Customers
 \item Developers
 \item Employees
 \item Journalists
 \item Investors
 \item Policymakers
 \item Suppliers
 \item Students
 \item Tradeshow Attendees
\end{itemize}

Let's try exercises using data that could be collected through surveys.

\section{Summarizing Data}

\subsection{Averages and Quartiles}
Averages and quartiles are a simple way to capture a snapshot of the data you have collected. Recall there are two types of data (continuous and categorical), so R will generate two types of sets of summary statistics.

The exercises in this section will use the "warpbreaks" dataset (The Number of Breaks in Yarn during Weaving) that is included with R. Load this data set into R following the instructions outlined in the Installation section of this work.

First, view the data by entering the following command:

print(warpbreaks)

Or, click View data set in R-Commander in the center of the user interface window.

This yields the following results:

% copy/paste results here

This is a long dataset, let's understand the number we are dealing with by entering the following command:

nrow(warpbreaks)

This yields the following results:

[1] 54

Meaning there are 54 rows in the dataset, or 54 samples of yarn in consideration. Consider the context. The dataset includes three variables, shown as column headers\footnote{Beyond looking at column headers, recall that you can have R return the variable names by entering: names(warpbreaks)}:

\begin{enumerate}
 \item Breaks - The number of reported breaks for each sample of yarn.
 \item Wool - The type of wool for each sample of yarn. A categorial variable, either A or B.
 \item Tension - The level of tension for each sample of yard. A categorical variable, either L (low), M (medium), or H (high).
\end{enumerate}

Assume further that yarn breaking during weaving is problematic. In other words, the higher the number in the Breaks column for each row, the worse that sample of yarn has performed. 

Second, calculate summary statistics by entering the following command:

summary(warpbreaks)

Or, select in R-Commander by going to Statistics > Summaries > Active Dataset.

This yields the following results:

% copy/paste results here

What is the significance of these results? Let's break them down one-by-one:

\begin{itemize}
 \item Min. - The minimum data point in the dataset. For column "breaks" this number is 10. In other words, of the surveyed samples of yarn, the lowest number of observed breaks in the dataset is 10. Stated differently, the best performing yarn had 10 breaks.
 \item 1st Qu. - The first quartile of data in the dataset. For column "breaks" this number is 18.25. In other words, of the surveyed samples of yarn, the bottom quarter of yarn surveyed had 18.25 breaks.
 \item Median - The median datapoint in the dataset. For column "breaks" this number is 26. In other words, of the surveyed samples of yarn, the middle data point is 26. This is calculated by cross-secting the data to the middle and selecting the single point, or adding the two middle points and dividing by two if the dataset has an even number. Median calculations are especially valuable for datasets that have outliers that may be skewing mean averages (more on this below).
 \item Mean - The mean datapoint for the dataset. For column "breaks" this number is 28.15. This is calculated by summing all of the data points and dividing by the total number (or count) of data points. Mean is what most people are referring to when they use the term average, and is the most common form of average used.
 \item 3rd Qu. - The third quartile of data in the dataset. For column "breaks" this number is 34. In other words, of the surveyed samples of yarn, the top quarter of yarn surveyed had 34 breaks.
 \item Max. - The maximum data point in the dataset. For column "breaks" this number is 70. In other words, of the surveyed samples of yarn, the highest number of observed breaks in the dataset is 70. Stated differently, the worst performing yarn had 70 breaks.
\end{itemize}

What else can we conclude from summary statistics of the "breaks" column? The mean is larger than the median, indicating a possible skew in the data towards the samples of yarn with higher numbers of breaks.

There are two other columns of summary statistics calculated by R: wool and tension. This provides a count summary of these categorical variables. Note that the dataset includes an even number of both types of wool (27 of both A and B). The dataset also includes an even number of all three levels of tension (18 Low, Medium, and High).

There is a third less common measure of average called mode\footnote{In R, mode is more importantly an object characteristic in indicating how the object is stored in memory (e.g. as a number, as a character string, as a function).}, the value that has the highest number of occurrences in the dataset. Calculating mode takes several steps, outlined below.

% Fill this out later

\section{Proportion Tests}
Use proportion tests to compare the proportion or mean from one group to a specified value. These will generate a p-value that can be used to evaluate the statistical significance of the comparison.\footnote{Learn more at: \texttt{https://stat.ethz.ch/R-manual/R-devel/library/stats/html/prop.test.html}.}

The exercises in this section will use the "warpbreaks" dataset (The Number of Breaks in Yarn during Weaving) that is included with R. Load this data set into R following the instructions outlined in the Installation section of this work.

\subsection{One Sample Proportion Test}
% This is temporary, crufty data that needs to be re-written
prop.test(<count>, <total>)
prop.test(384,5000)
prop.test(384, 5000, conf.level=0.95)

\subsection{Two Sample Proportion Test}
% This is temporary, crufty data that needs to be re-written
prop.test(x = c(388, 545), n = c(443, 564), correct = FALSE)

2-sample test for equality of proportions without continuity
        correction

data:  c(388, 545) out of c(443, 564)
X-squared = 29.824, df = 1, p-value = 4.731e-08
alternative hypothesis: two.sided
95 percent confidence interval:
 -0.12459255 -0.05633856
sample estimates:
   prop 1    prop 2 
0.8758465 0.9663121 

\section{Generating Metadata from Open Responses}
% Section is incomplete, needs a screen shot to demonstrate how
It is possible to create metadata from open-ended responses in survey questions that you evaluate through statistical techniques.

For example, say a survey asked customers if there is anything your organization could do to better serve them in the future. Have someone from your organization review the responses and generate a list based on topics submitted by customers. 

Next, count each time on appears in the customer responses. This can be done manually or through a formula in spreadsheet software like LibreOffice Calc or Gnumeric.\footnote{Learn more about these projects at \texttt{https://www.libreoffice.org/} and \texttt{http://gnumeric.org/}} This new set of metadata can now be leveraged in proportion tests or other testing methods.