%
% Installation.tex
% Installation
%
% R for Business Administration
%
% Copyright (C) 2016 Harris Kenny, Brandon DeGolier, Nate Lewis
%
% This document is licensed under the Creative Commons Attribution 4.0
% International Public License (CC BY-SA 4.0)
%

\section{Your Computing Environment}
R is a Free Software\footnote{Free software is software that gives you the user the freedom to share, study and modify it. We call this free software because the user is free. Learn more at \texttt{http://www.fsf.org/about/what-is-free-software}.} project shared under the terms of the Free Software Foundation's GNU General Public License. Similarly, this work recommends you use GNU/Linux for your computing environment. 

Not sure where to start? Learn more about some of the leading GNU/Linux distributions:

\begin{itemize}
 \item Debian: \texttt{http://www.debian.org/}
 \item Fedora: \texttt{https://getfedora.org/}
 \item Ubuntu: \texttt{http://www.ubuntu.com/}
\end{itemize}

\section{Installing R}
R is a software environment that is capable of data manipulation, calculation, and graphical display. The term "environment" is used to communicate that it is a fully planned and coherent system, within which statistical techniques are implemented.

R is an implementation of the S programming language, created by John Chambers while at Bell Labs, combined with lexical scoping semantics inspired by Scheme. R was created by Ross Ihaka and Robert Gentleman at the University of Auckland, New Zealand and an initial version was released in 1994.\footnote{Learn more at: \texttt{https://www.r-project.org/}.}

To install R:

\begin{enumerate}
 \item Go to \texttt{https://cran.r-project.org/mirrors.html}.
 \item Select a location closest to you.
 \item Click on your operating system and follow the appropriate directions.
\end{enumerate}

R is part of many GNU/Linux distributions, you should check with your GNU/Linux package management system in addition to the link above.

\section{Installing Sample Datasets}
R has an included package with sample datasets that are used throughout this work to demonstrate concepts and applications for R. This work will teach you various techniques using this data, that you can then apply to your own data.

To load a sample dataset, for example named foobar, enter the following command in R Console:

data(foobar)

You could also enter:

data("foobar")

Note that these are case sensitive.\foonote{For more information on how the data command works, visit: \texttt{https://stat.ethz.ch/R-manual/R-devel/library/utils/html/data.html}.}

\section{Installing R Commander}
R has a fairly high learning threshold; it requires learning a programming language and provides relatively little visual feedback. Thankfully, there is a freely licensed tool called R-Commander which helps solve both of these problems by creating a graphical interface for R.

Combined, they make a compelling solution for those who only need basic analysis, while also allowing greater configuration and depth for those who need more sophisticated analysis.\footnote{Learn more at: \texttt{http://socserv.mcmaster.ca/jfox/Misc/Rcmdr/}.}

To install R-Commander:

\begin{enumerate}
 \item Once you have installed R, open it by double-clicking on the icon or opening through a terminal emulator.
 \item A window called “R Console” will open.
 \item At the prompt (the > symbol), type the following command exactly and then press enter:

install.packages("Rcmdr", dependencies = TRUE)

 \item R should respond by asking you to select a mirror site, and listing them in a pop-up box.
 \item Choose a nearby location.
 \item Depending on your connection speed, the installation may take a while. Be patient and wait until you see the prompt again before you do anything.
 \end{enumerate}
 
 \section{Starting R Commander}
If R is not already open, open it by clicking on its icon or through a terminal emulator. To open R Commander, at the prompt enter the following command:

library(Rcmdr)

You should see a large new window pop up, labeled R Commander.\footnote{Find detailed installation instructions online: \texttt{http://socserv.mcmaster.ca/jfox/Misc/Rcmdr/installation-notes.html}} You are now ready to analyze your data with R Commander. If you close this window while R is still open, you can start R Commander again by entering the command Commander() in R Console. Entering library(Rcmdr) in this situation will not work unless you close R and open it again.

Congratulations! Now let's get to work.