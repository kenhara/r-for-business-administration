%
% Introduction.tex
% Introduction
%
% R for Business Administration
%
% Copyright (C) 2016 Harris Kenny
%
% This document is licensed under the Creative Commons Attribution 4.0
% International Public License (CC BY-SA 4.0)
% 
% Additional resources: https://web.princeton.edu/sites/opplab/papers/Diemand-Yauman_Oppenheimer_2010.pdf
% http://www.stern.nyu.edu/faculty/bio/adam-alter
% http://fivethirtyeight.com/features/charles-duhigg-data-can-help-us-change-but-we-still-have-to-do-the-hard-work/

\subsection{What This Is (And Is Not)}
As defined on Wikipedia, "Data is a set of values of qualitative or 
quantitative variables; restated, pieces of data are individual pieces of 
information." The article continues, "Data is the least abstract, information 
the next least, and knowledge the most.

Data becomes information by interpretation; e.g., the height of Mt. Everest is 
generally considered 'data', a book on Mt. Everest geological characteristics 
may be considered 'information', and a report containing practical information 
on the best way to reach Mt. Everest's peak may be considered 'knowledge'...

Some complement the series 'data', 'information' and 'knowledge' with 'wisdom',
 which would mean the status of a person in possession of a certain 'knowledge'
  who also knows under which circumstances is good to use it."\footnote{Source: \url{https://en.wikipedia.org/w/index.php?title=Data&oldid=733102554}}

This work strives to be a practical guide to using R for business 
administration. In other words, to teach business professionals and students 
how to use R to convert data into information.

This work cannot provide the knowledge on the best way to release a new 
product, market to customers, etc. This work also cannot provide the wisdom to 
understand the depths of applying the field of statistics to business 
decision-making. Instead, this work strives to be a starting point to catalyze 
further inquiry.

Many others have written helpful books, guides, blog posts, videos, articles, 
and tutorials on statistics, using R, communicating with numbers, and more. 
This a complementary effort and if using the techniques outlined herein is 
valuable, you will most likely require additional resources to get the most 
value for your organization.

The following example from Cincinnati, Ohio, USA will hopefully illustrate this
 point, which I learned about through an episode of FiveThirtyEight's What's 
 the Point podcast entitled, "Charles Duhigg: Data Can Help Us Change, But We 
 Still Have To Do The Hard Work" hosted by Jody Avirgan. 

\newpage

\subsection{Cognitive Disfluency in Cincinnati}
This anecdote is included to demonstrate an example when data alone was 
insufficient in answering an organization's questions.\footnote{This story is 
summarized from Charles Duhigg's 2016 book Smarter Better Faster: The Secrets 
of Being Productive in Life and Business.} Similarly, data and the techniques 
outlined in this work are unlikely to answer your organization's questions.

In 2007, South Avondale Elementary School was ranked as one of the worst 
schools in Cincinnati, Ohio, USA. Situated in a low income neighborhood with 
high unemployment and crime; students, parents, teachers, and administrators 
were facing difficult odds and the school had been declared an academic 
emergency. 

In response, officials and corporate partners invested heavily to be on the 
leading edge of using data and technology to improve performance. Using a 
variety of tools, data on things like attendance, homework, test scores, and 
participation were all rolled up into memoranda, dashboards, and reports 
detailing a variety of metrics.

The experiment was failing, with 90 percent of educators admitting they did not
 review the data being collected on their students and sent to them. In 
 response, they created the Elementary Initiative, which improved performance 
 from only 37 percent of students meeting education standards to over 80 
 percent of students.

This program leveraged a concept known as cognitive disfluency by having 
teachers use a literal data room to physically meet and manually transcribe 
data, draw graphs, and plan classroom experiments. By disrupting their 
established pattern of consuming information through dashboards, the teachers 
developed a deeper understanding of the data and were able to leverage insights
 in successful ways.

The takeaway is not that you must replicate this process for your organization 
(although it may be helpful to do so). Instead, recognize that integrating data
 into decision-making for your organization is a process that will take time, 
 effort, trial, and error.

Given the challenge ahead, consider imitating successful Free Software and Open
 Source Hardware communities by sharing your questions, answers, and lessons 
 learned with others.

Thank you,

Harris Kenny