%
% Introduction.tex
% Introduction
%
% R for Business Administration
%
% Copyright (C) 2016 Harris Kenny, Brandon DeGolier, Nate Lewis
%
% This document is licensed under the Creative Commons Attribution 4.0
% International Public License (CC BY-SA 4.0)
%

\subsection{Purpose of this Work}
This work strives to be a practical guide to using R for business 
administration. While this work contains information about using statistics in business, it is no substitute for learning statistics.

\subsection{Data, Information, Knowledge, and Wisdom}
As defined on Wikipedia, "Data is a set of values of qualitative or quantitative variables; restated, pieces of data are individual pieces of information." The article continues, "Data is the least abstract, information the next least, and knowledge the most.

Data becomes information by interpretation; e.g., the height of Mt. Everest is generally considered "data", a book on Mt. Everest geological characteristics may be considered "information", and a report containing practical information on the best way to reach Mt. Everest's peak may be considered "knowledge".

Some complement the series "data", "information" and "knowledge" with "wisdom", which would mean the status of a person in possession of a certain "knowledge" who also knows under which circumstances is good to use it."\footnote{Source: \texttt{https://en.wikipedia.org/w/index.php?title=Data&oldid=733102554}.}

\subsection{Using and Misusing Data}
