%
% Introduction.tex
% Introduction
%
% R for Business Administration
%
% Copyright (C) 2016 Harris Kenny
%
% This document is licensed under the Creative Commons Attribution 4.0
% International Public License (CC BY-SA 4.0)
%

As defined on Wikipedia, "Data is a set of values of qualitative or quantitative variables; restated, pieces of data are individual pieces of information." The article continues, "Data is the least abstract, information the next least, and knowledge the most.

Data becomes information by interpretation; e.g., the height of Mt. Everest is generally considered "data", a book on Mt. Everest geological characteristics may be considered "information", and a report containing practical information on the best way to reach Mt. Everest's peak may be considered "knowledge"...

Some complement the series "data", "information" and "knowledge" with "wisdom", which would mean the status of a person in possession of a certain "knowledge" who also knows under which circumstances is good to use it."\footnote{Source: \texttt{https://en.wikipedia.org/w/index.php?title=Data&oldid=733102554}.}

This work strives to be a practical guide to using R for business 
administration. In other words, to teach business professionals and students how to use R to convert data into information.

This work cannot provide the knowledge on the best way to release a new product, market to customers, etc. This work also cannot provide the wisdom to understand the depths of applying the field of statistics to business decision-making. Instead, this work strives to be a starting point to catalyze further inquiry.

Many others have written helpful books, guides, blog posts, videos, articles, and tutorials on statistics, using R, communicating with numbers, and more. This a complementary effort and if using the techniques outlined herein is valuable, you will most likely require additional resources to get the most value for your organization.

The following example from Cincinnati, Ohio, USA will hopefully illustrate this point, which I learned about through an episode of FiveThirtyEight's What's the Point podcast entitled "Charles Duhigg: Data Can Help Us Change, But We Still Have To Do The Hard Work," hosted by Jody Avirgan.\footnote{Learn more at: \texttt{http://fivethirtyeight.com/features/charles-duhigg-data-can-help-us-change-but-we-still-have-to-do-the-hard-work/}.} 


