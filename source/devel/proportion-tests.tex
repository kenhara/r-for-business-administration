\section{Proportion Tests}
Use proportion tests to compare the proportion or mean from one group to a specified value. These will generate a p-value that can be used to evaluate the statistical significance of the comparison.\footnote{Learn more at: \texttt{https://stat.ethz.ch/R-manual/R-devel/library/stats/html/prop.test.html}.}

The exercises in this section will use the "warpbreaks" dataset (The Number of Breaks in Yarn during Weaving) that is included with R. Load this data set into R following the instructions outlined in the Installation section of this work.

\subsection{One Sample Proportion Test}
% This is temporary, crufty data that needs to be re-written
prop.test(<count>, <total>)
prop.test(384,5000)
prop.test(384, 5000, conf.level=0.95)

\subsection{Two Sample Proportion Test}
% This is temporary, crufty data that needs to be re-written
prop.test(x = c(388, 545), n = c(443, 564), correct = FALSE)

2-sample test for equality of proportions without continuity
        correction

data:  c(388, 545) out of c(443, 564)
X-squared = 29.824, df = 1, p-value = 4.731e-08
alternative hypothesis: two.sided
95 percent confidence interval:
 -0.12459255 -0.05633856
sample estimates:
   prop 1    prop 2 
0.8758465 0.9663121 