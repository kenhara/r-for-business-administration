%
% Purchasing.tex
%
% Aleph Objects Operations Manual
%
% Copyright (C) 2014 Aleph Objects, Inc.
%
% This document is licensed under the Creative Commons Attribution 4.0
% International Public License (CC BY-SA 4.0) by Aleph Objects, Inc.
%


\section{Requester}

\begin{enumerate}
\item Post the product request to the Product Requests Group.
  \begin{enumerate}
  \item From messaging, pick Product Requests from My Groups.
  \item Click on the "send a message to the group" box.
  \item Click on the icon on the top right corner of the message box with a hover info “Open the full mail composer”
  \item Pick the “Internal Product Request” in the use template pick (bottom right corner of the popup)
  \item Type the product request with the following information
    \begin{itemize}
    \item Date Needed:
    \item Name of Requester:
    \item Internal Charge account code for PO:
    \item Item description:
    \item Quantity:
    \item Estimated Cost:
    \item Suggested supplier:
    \item Supplier model/item number:
    \item Purpose: 
    \end{itemize}
  \item Click on the send button
  \end{enumerate}
Note: The product requests group by default includes all employees. The requester is any Aleph Objects employee. To make the request, the employee needs an OpenERP account.
\item If the materials planner or purchase approver requires additional information, they may reply to this message and it will show up in the requester's Inbox.
\end{enumerate}

\section{Materials Planner}

This role creates the Purchase Quotation. The following steps are to be followed to generate a purchase quotation.

\begin{enumerate}
\item Review the Purchase Requests message group for new product requests. 
\item Note the details of the request and identify the product from the OpenERP product catalog -- this is done by searching for the product from Purchases/Product/Product menu. This will only show products that have been marked "Can be purchased."
\item Create a new purchase quotation from Purchases/Purchase/Quotations.
\item Pick the supplier (this can be reviewed from the supplier list on the product identified in step 2)
\item Add supplier reference and source document.
\item Pick the product in the order lines and set the quantity and adjust unit price if it is different. 
\item Add terms and conditions, payment terms, expected date to the RFQ.
\item Save RFQ.
\item Confirm the purchase order. If the order amount is less than 100, the order will be approved automatically. If not, the Purchase Order will wait for a second approval. Refer to Purchase Approver process before continuing with the next steps.
\item There are two types of purchase orders possible: The first is when the materials planner goes ahead and makes the purchase on his/her credit card and/or gets the material for immediate use. The second is when a formal order is placed with a supplier and the supplies / supplier invoice are awaited.
\item An incoming shipment document and a draft supplier invoice will be created on PO confirmation.
\item Option A: Purchase on Credit Card (Prepaid purchases)
  \begin{enumerate}
  \item The materials planner goes ahead and makes the purchase of the required materials and makes a payment using the credit/debit card. [Going to the nearest store or placing an order on the supplier website]
  \item If materials are purchased from a local store, the materials planner updates the receipt of goods in OpenERP. If purchase order is placed on the supplier website, the incoming shipment will be handled by the warehouse person on its arrival.
    \begin{enumerate}
    \item Go to Warehouse/ Receive/Deliver by Orders /Incoming Shipments.
    \item Select the incoming shipment document corresponding to the PO.
    \item Mark the goods as received.
    \item Create draft supplier invoice from incoming shipment / purchase order.
    \end{enumerate}
  \item Turn in the details of the payment and the PO reference number to Accounting.  Accounting Person continues with the workflow.
  \end{enumerate}
\item Option B: Placing purchase orders by email
  \begin{enumerate}
  \item The materials planner confirms the purchase order. He/she then emails/faxes the PO to the supplier. 
  \item The warehouse person processes material receipt
    \begin{enumerate}
    \item Go to Warehouse/ Receive/Deliver by Orders /Incoming Shipments.
    \item Select the incoming shipment document corresponding to the PO.
    \item Mark the goods as received.
    \item Create draft supplier invoice from incoming shipment / purchase order.
    \end{enumerate}
  \item Accounting Person continues with the workflow.
  \end{enumerate}
\end{enumerate}

\section{Purchase Approver}

This role approves the RFQ by confirming it to become a purchase order. The approval step may be carried out by the Materials Planner based on the purchase amount rule [amount > 100].

\begin{enumerate}
\item The approver lists all Purchase Orders that are in “Waiting Approval” state.
\item The details of the PO are reviewed.
\item The approver can then confirm the PO  to approve or set it back to “draft” state” to reject.
\item The materials planner will be notified for the approval state.
\end{enumerate}

\section{Accounting}

This role manages the supplier invoice and payments to suppliers.
\begin{enumerate}
\item The warehouse person on the materials planning person has already created a draft supplier invoice.
\item On receipt of the incoming shipment note (Option B)/payment made (Option A), the accountant/book keeper adjusts the draft invoice and confirms the invoice.
\item The invoice is now ready for payment. The payment can be processed in a few ways:
  \begin{enumerate}
  \item Click on Register Payment in the supplier invoice screen. Follow through on the wizard that pops up and record the payment. This method requires the actual check/payment processing to happen outside of OpenERP. The payment made will be adjusted with the current invoice even if there are pending payments to the supplier.
  \item Click on Accounting/Supplier/Supplier Payment menu and create a new payment. Select the supplier. All pending payments will be listed and payment amount will be adjusted against the pending payments including the current one based on the amount and payment aging. This method is relevant and useful when payments to suppliers are batched.
  \item Payment through checks generated in OpenERP. This section will be elaborated after the installation and configuration of check writing module. 
  \end{enumerate}
\end{enumerate}

\section{Notes}

\begin{enumerate}
\item Record conversations with the supplier during the negotiation if any along with the purchase order using “Log a Note” feature. If the information needs to reach the supplier then the “Send Message” option should be used.
\item Once the purchase order is confirmed the only changes allowed are to the text fields. The supplier, pricing details, product details are frozen and cannot be changed. Please review all the options before confirming the PO.
\end{enumerate}

\section{Products}
A Product in OpenERP language, is anything you buy or sell.

OpenERP ---> Purchases ---> Products ---> Products

\section{Suppliers}
A supplier is a Partner that we buy Products from.

OpenERP ---> Purchases ---> Purchase ---> Supplier

\section{Inventory}

\subsection{Receiving}
Keep track of what comes in, count it, track in OpenERP:

OpenERP ---> Purchases ---> Incoming Products ---> Incoming Shipments

OpenERP ---> Purchases ---> Incoming Products ---> Incoming Products

\subsection{Moves}

OpenERP ---> Warehouse ---> Traceability ---> Stock Moves


